\chapter*{Závěr}
\addcontentsline{toc}{chapter}{Závěr}

V~této práci bylo implementováno rozšíření do webových prohlížečů pro zjednodušení práce uživatelů integrovaného studijního informačního systému InSIS.

Byly navrženy 3 nové funkcionality, které byly následně implementovány. První přidanou funkcionalitou je náhled rozvrhu při registraci rozvrhových akcí, který zvyšuje produktivitu uživatelů při zápisu předmětů. Další přidanou funkcionalitou je správa a zasílání připomenutí na odevzdávárny s blížícím se datem odevzdání. Poslední implementovanou funkcionalitou je vylepšená verze zobrazení rozvrhu, která oproti původní verzi přidává možnost přidávání poznámek ke konkrétním hodinám, přepínání mezi výukovými týdny nebo automatické zpracování volných dní a konání blokových akcí.

První cíl práce, kterým byl návrh a následná implementace rozšíření byl splněn. Sestavené rozšíření do webových prohlížečů bylo publikováno na internetové obchody Google Web Store a Firefox Addons a v současnosti má přes 300 instalací a 100 aktivních uživatelů. Podpůrný webový server byl implementován, otestován a nasazen na platformě Digital Ocean.

Druhý cíl práce, kterým bylo porovnání dostupných technologií a výběr vhodné technologie pro implementaci obou částí softwarového řešení. Kapitola \ref{chap:volba-technologii} představuje jednotlivé dostupné technologie a jejich srovnání v kontextu implementace. 

Poslední cíl práce, kterým byla evaluace zpětné vazby od ostatních studentů, byl také splněn. V kapitole 
\ref{chap:zpetna-vazba} byly potvrzeny 2 ze 3 stanovených hypotéz. Zároveň byly v rámci dotazníkového šetření sesbírány podklady pro další zlepšení webového rozšíření, které je možné do budoucna realizovat.