\chapter*{Závěr}
\addcontentsline{toc}{chapter}{Závěr}

Předmětem této práce bylo navržení a následná implementace rozšíření do webových prohlížečů pro zjednodušení práce a zvýšení efektivity uživatelů integrovaného studijního informačního systému InSIS.

V~práci byly navrženy 3 nové funkcionality, které byly následně v~podobě rozšíření implementovány. První přidanou funkcionalitou je náhled rozvrhu při registraci rozvrhových akcí. Další přidanou funkcionalitou je správa a zasílání připomenutí na odevzdávárny s~blížícím se datem odevzdání, což bylo velice žádané rozšíření již existujícího modulu informačního systému InSIS. Poslední implementovanou funkcionalitou je vylepšená verze zobrazení aktuálního rozvrhu studenta, která oproti původní verzi přidává možnost přidávání poznámek ke konkrétním hodinám, přepínání mezi výukovými týdny nebo automatické zpracování volných dní a konání blokových akcí.

Prvním cílem práce byl návrh a implementace rozšíření do webových prohlížečů. Kapitola \ref{chap:navrh-a-specifikace} se zabývá návrhem výše zmíněných přidaných funkcionalit. V~kapitolách \ref{chap:server} a \ref{chap:extension} je detailně popsán vývoj obou dílčích částí softwarového řešení. První z~těchto částí je podpůrný webový server, který byl implementován, otestován a nasazen na platformě Digital Ocean. Druhá část, tedy samotné rozšíření do webových prohlížečů je navržené tak, aby bylo snadné do budoucna přidávat další funkcionality nebo měnit současnou podobu přidaných funkcionalit. Podpůrný webový server je distribuován v~podobě Docker kontejneru, který je možné snadno přenášet mezi hostingovými platformami v~případě změny požadavků na dostupný výpočetní výkon. 

Druhým cílem práce bylo porovnání a výběr technologií pro implementaci obou částí řešení. Kapitola \ref{chap:volba-technologii} představuje některé z~dostupných technologií a jejich srovnání v~příslušném kontextu. Pro vývoj rozšíření do webových prohlížečů byl zvolen programovací jazyk TypeScript a knihovna React.js společně se sadou komponent Chakra UI. Pro podpůrný webový server byl zvolen aplikační rámec Spring s~nadstavbou Spring Boot společně s~programovacím jazykem Kotlin a systémem řízení báze dat PostgreSQL. 

Poslední cílem práce byla evaluace zpětné vazby od ostatních studentů, kteří rozšíření používají. V~kapitole \ref{chap:zpetna-vazba} bylo vyhodnoceno dotazníkové šetření, které odpovídalo na 3 stanovené výzkumné otázky. Návrhu rozšíření předcházela analýza používání systému InSIS jeho uživateli. Účelem výzkumných otázek stanovených v~kapitole \ref{chap:zpetna-vazba} je vyhodnotit míru shody této počáteční analýzy s~používáním výslené podoby rozšíření. Provedené dotazníkové šetření zároveň poskytuje autorovi přímou zpětnou vazbu v~podobě odpovědí na kvalitativní otázky.

Všechny vymezené cíle práce byly splněny a výsledkem práce je funkční rozšíření do webových prohlížečů, které je publikováno na internetových obchodech Google Web Store a Firefox Addons. V~době psaní má rozšíření do prohlížečů přes 300 instalací a více než 120 aktivních uživatelů. 

\clearpage
Do budoucna by bylo možné na tuto práci navázat především přidáním podpory pro další webové prohlížeče, zejména Apple Safari, který je výchozím webovým prohlížečem v~operačním systému macOS, který řada studentů používá. Dále by bylo možné přidat dokumentaci a uživatelský manuál pro zjednodušení orientace nových uživatelů v~přidaných funkcionalitách. Před přidáním dalších funkcionalit do rozšíření by bylo vhodné rozšířit povědomí o~existenci rozšíření a tím zvýšit počet jeho uživatelů a tedy i vzorku respondentů pro budoucí dotazníková šetření. 
