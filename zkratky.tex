\chapter*{\SeznamZkratek}

\begin{multicols}{2}
\raggedright
\begin{description}
% \item [BCC] Blind Carbon Copy
% \item [CC] Carbon Copy
% \item [CERT] Computer Emergency Response Team
% \item [CSS] Cascading Styleheets
% \item [DOI] Digital Object Identifier
% \item [REST] Representational State Transfer
% \item [SOAP] Simple Object Access Protocol
% \item [URI] Uniform Resource Identifier
% \item [URL] Uniform Resource Locator
\item [XML] eXtended Markup Language
\item [JSON] JavaScript Object Notation
\item [HTTP] Hyper Text Transfer Protocol
\item [DTO] Data Transfer Object
\item [URL] Uniform Resource Locator
\item [REST] Representational State Transfer
\item [CRUD] Create, Read, Update, Delete
\item [HTML] Hypertext Markup Language
\item [SQL] Structured Query Language
\item [AJAX] Asynchronous JavaScript and XML
\item [API] Application Programming Interface
\item [JVM] Java Virtual Machine
\item [CLR] Common Language Runtime
\item [NPM] Node Package Manager
\item [UI] User Interface
\item [UX] User Experience
\item [CI] Continuous Integration
\item [CD] Continuous Delivery
\end{description}
\end{multicols}

% Poznámka: Seznam zkratek je vhodný použít, pokud počet zkratek v textu práce je větší než 20 a nejedná se o zkratky běžné.