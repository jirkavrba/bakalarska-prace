\chapter*{Úvod}
\addcontentsline{toc}{chapter}{Úvod}

Během svého studia na Vysoké škole ekonomické v~Praze všichni studenti přijdou do styku se školním informačním systémem InSIS. Tento informační systém obsahuje širokou škálu modulů a funkcionalit, se kterými studenti během svého studia na vysoké škole pracují.

Ačkoliv tento systém poskytuje svým uživatelům všechny potřebné informace, přidáním některých funkcionalit by bylo možné zlepšit produktivitu uživatelů tohoto systému a celkově zpříjemnit studium studentů.

Tato práce si klade za cíl vymezit, a následně implementovat rozšiřující funkcionality do informačního systému InSIS, které následně budou distribuovány v~podobě rozšíření do webových prohlížečů a podpůrného webového serveru, který poskytuje REST API pro uživatele rozšíření. Dílčím cílem práce je pak popsat výběr technologií použité pro implementace zmíněného webového rozšíření a podpůrného webového serveru.

Motivací pro tuto práci byla negativní zkušenost autora a jeho spolužáků s~používáním některých modulů školního informačního systému InSIS. 
Předchůdcem této práce byla sada uživatelských skriptů napsaných autorem v~průběhu studia, které má tato práce za cíl nahradit a naimplementovat nové, rozšiřující funkcionality do informačního systému InSIS v~podobě koherentního softwarového řešení, které bude jednoduché na instalaci a používání.

Tato práce je dělena do celkem \textbf{pěti částí}: 

\begin{enumerate}
    \item Návrh a popisem implementovaných funkcionalit,
    \item Proces výběru technologií použitých pro implementaci,
    \item Implementace podpůrného webového serveru
    \item Implementace rozšíření do webových prohlížečů
    \item Evaluace zpětné vazby od ostatních studentů, kteří webové rozšíření používají
\end{enumerate}

Práce se zaměřuje především na praktickou problematiku spojenou s~implementací softwarového řešení s~výjimkou první a poslední části, které jsou zaměřené více teoreticky.

\todo{Je dobrý tady mít konkrétní čísla?}

Jedním z~výsledků této práce je softwarový produkt, který v~době psaní čítá 238 instalací z~internetového obchodu Google Web Store a 43 instalací z~internetového obchodu Firefox Addons.

\todo{Dopsat úvod}

% Úvod je povinnou částí bakalářské/diplomové práce. Úvod je uvedením do tématu. Zvolené téma rozvádí, stručně ho zasazuje do souvislostí (může zde být i popis motivace k sepsání práce) a odpovídá na otázku, proč bylo téma zvoleno. Zasazuje téma do souvislostí a zdůvodňuje jeho nutnost a aktuálnost řešení. Obsahuje explicitně uvedený cíl práce. Text cíle práce je shodný s textem, který je uveden v zadání bakalářské práce, tj. s textem, který je uveden v systému InSIS a který je také uveden v části Abstrakt.

% Součástí úvodu je také stručné představení postupu zpracování práce (detailně je metodě zpracování věnována samostatná část vlastního textu práce). Úvod může zahrnovat i popis motivace k sepsání práce.

% Úvod k diplomové práci musí být propracovanější -- podrobněji to je uvedeno v Náležitostech diplomové práce v rámci Intranetu pro studenty FIS.

% Následuje několik ukázkových kapitol, které doporučují, jak by se měla bakalářská/diplomová práce sázet. Primárně popisují použití \TeX{}ové šablony, ale obecné rady poslouží dobře i~uživatelům jiných systémů.

