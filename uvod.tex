\chapter*{Úvod}
\addcontentsline{toc}{chapter}{Úvod}

Během svého studia na Vysoké škole ekonomické v~Praze všichni studenti přijdou do styku se školním informačním systémem InSIS. Tento informační systém obsahuje širokou škálu modulů a funkcionalit, se kterými studenti během svého studia na vysoké škole pracují. Tento systém se skládá z velkého množství samostaných modulů, které obsahují nespočet 

Ačkoliv tento systém poskytuje svým uživatelům všechny potřebné informace, přidáním některých funkcionalit by bylo možné zlepšit produktivitu uživatelů tohoto systému a celkově zpříjemnit studium studentů.

Tato práce si klade za cíl vymezit a následně implementovat rozšiřující funkcionality do informačního systému InSIS, které následně budou distribuovány v~podobě rozšíření do webových prohlížečů. Součástí práce je zároveň implementace podpůrného webového serveru, který poskytuje REST API pro uživatele rozšíření. Dále práce popisuje výběr použitých technologií pro implementací obou softwarových částí produktu. Posledním cílem je evaluace zpětné vazby od ostatních studentů a ověření stanovených hypotéz.

Motivací pro tuto práci byla negativní zkušenost autora a jeho spolužáků s~používáním některých modulů školního informačního systému InSIS. 
Současné podobě této práce předcházela sada uživatelských skriptů naprogramovaných autorem v~průběhu studia, které má tato práce za cíl nahradit a naimplementovat nové, rozšiřující funkcionality do informačního systému InSIS v~podobě koherentního softwarového řešení, které bude jednoduché na instalaci a používání.

Práce je dělena do celkem pěti částí: 

\begin{enumerate}
    \item Návrh a popisem implementovaných funkcionalit,
    \item Proces výběru technologií použitých pro implementaci,
    \item Implementace podpůrného webového serveru
    \item Implementace rozšíření do webových prohlížečů
    \item Evaluace zpětné vazby od ostatních studentů, kteří webové rozšíření používají
\end{enumerate}

Práce se zaměřuje především na praktickou problematiku spojenou s~implementací softwarového řešení s~výjimkou první a poslední části, které jsou zaměřené více teoreticky.

Výsledkem práce je webové rozšíření pro prohlížeče založené na technologii Google Chrome a Mozilla Firefox, publikované na internetové obchody Google Play a Firefox Addons. 
